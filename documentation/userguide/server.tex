\documentclass[notoc,justified]{tufte-book}    % `notoc` suppresses TL custom TOC, reverts to standard LaTeX
\usepackage{graphicx}
\hyphenation{HMMER}
\newcommand{\hmmserver}{\mono{hmmserver}}
\newcommand{\Hmmserver}{\mono{Hmmserver}}
\newcommand{\hmmclient}{\mono{hmmclient}}
\newcommand{\Hmmclient}{\mono{Hmmclient}}
\newcommand{\hmmpgmd}{\mono{hmmpgmd}}
\newcommand{\userguide}{HMMER User's Guide}
\title{User's Guide for HMMER's Server and Remote Client Programs}

\subtitle{High-performance biological sequence analysis using profile hidden Markov models}

\author{Nicholas P. Carter, Sean R. Eddy}
\subauthor{and the HMMER development team}

\pkgurl{http://hmmer.org}
\pkgversion{3.3.2}   % ./configure replaces HMMER_VERSION
\pkgdate{Nov 2020}         %    ... and HMMER_DATE

                    % definitions for \maketitle 
\bibliographystyle{unsrtnat-brief}   % customized natbib unsrtnat. Abbrev 3+ authors to ``et al.'' 

\begin{document}
\setcounter{tocdepth}{2}             % 0=chapters 1=sections 2=subsections 3=subsubsections? 4=paragraphs
\newcommand{\UNIrelease}{2020\_05}
\newcommand{\UNInseq}{563,552}

\newcommand{\HMMERversion}{3.3.2}
\newcommand{\HMMERdate}{Nov 2020}

\newcommand{\BGLnseq}{4}
\newcommand{\BGLalen}{171}
\newcommand{\BGLmlen}{149}
\newcommand{\BGLgaps}{22}
\newcommand{\BGLeffn}{0.96}
\newcommand{\BGLre}{0.589}

\newcommand{\HMMERfmtversion}{f}
\newcommand{\HMMERsavestamp}{[3.3.2 | Nov 2020]}

\newcommand{\SGUevalue}{4.9e-65}
\newcommand{\SGUbitscore}{223.2}
\newcommand{\SGUbias}{0.1}
\newcommand{\SGUorigscore}{223.3}
\newcommand{\SGUdombitscore}{223.0}
\newcommand{\SGUseqname}{HBB\_GORGO}
\newcommand{\SGUmsvpass}{3.7}
\newcommand{\SGUbiaspass}{17002}
\newcommand{\SGUvitpass}{2323}
\newcommand{\SGUfwdpass}{1129}
\newcommand{\SGUelapsed}{0.9}

\newcommand{\SFSevalue}{5.6e-57}
\newcommand{\SFSbitscore}{176.4}
\newcommand{\SFSdomevalue}{2.3e-16}
\newcommand{\SFSdombitscore}{46.2}
\newcommand{\SFSexpdom}{9.8}
\newcommand{\SFSndom}{9}

\newcommand{\SFSmaxdom}{7}
\newcommand{\SFSmaxdomu}{5}
\newcommand{\SFSmaxsc}{46.2}
\newcommand{\SFSievalue}{2.3e-16}
\newcommand{\SFSuievalue}{1.3e-10}
\newcommand{\SFSdomZ}{794}
\newcommand{\SFSucevalue}{1.9e-13}
\newcommand{\SFSaidx}{1}
\newcommand{\SFSascore}{-1.9}
\newcommand{\SFSaevalue}{0.24}
\newcommand{\SFSauevalue}{191}
\newcommand{\SFSacoords}{395-410}
\newcommand{\SFSbidx}{6}
\newcommand{\SFSbscore}{0.4}
\newcommand{\SFSbevalue}{0.045}
\newcommand{\SFSbuevalue}{35.7}
\newcommand{\SFSbcoords}{1742-1769}
\newcommand{\SFSainsig}{4.6}
\newcommand{\SFSbinsig}{8.9}

\newcommand{\JHUninc}{955}
\newcommand{\JHUnsig}{955}

\newcommand{\NMHafrom}{302390}
\newcommand{\NMHato}{302466}
\newcommand{\NMHbfrom}{302466}
\newcommand{\NMHbto}{302389}
\newcommand{\NMHnres}{660000}
\newcommand{\NMHntop}{330000}
\newcommand{\NMHnssv}{73493}
\newcommand{\NMHfracssv}{11.1}
\newcommand{\NMHnbias}{49311}
\newcommand{\NMHfracbias}{7.5}
\newcommand{\NMHnvit}{4022}
\newcommand{\NMHfracvit}{0.6}
\newcommand{\NMHnfwd}{1562}
    % snippets captured from output, by gen-inclusions.py 

\maketitle

\input{copyright}

\begin{adjustwidth}{}{-1in}          % TL \textwidth is quite narrow. Expand it manually for TOC and man pages.
\tableofcontents                     
\end{adjustwidth}


\chapter{Introduction}
\Hmmserver and \hmmclient are replacements for the \hmmpgmd and \mono{hmmc2} programs provided by earlier versions of HMMER.  Like \hmmpgmd, \hmmserver is a persistent, long-running, service that provides high-performance homology searches by caching sequence and HMM databases in RAM and distributing the work of each search across many computers and threads.  \Hmmclient is a full-featured command-line client application for \hmmserver that submits searches to a running server and displays results in a format that is as close as possible to that of the \mono{hmmsearch}, \mono{hmmscan}, \mono{phmmer}, and \mono{jackhmmer} programs.  It is thus a significant upgrade to the \mono{hmmc2} program, which was more of a debugging tool for \hmmpgmd than a full client program.



\chapter{Installation}
\Hmmserver and \hmmclient are included in the standard HMMER distribution package, but HMMER must be compiled with MPI support turned on for \hmmserver to be useful, so it is likely you will have to compile HMMER from source to use the server.  To do this, obtain a source-code copy of HMMER (see the \userguide for instructons on how to do this) and go through the standard configuration/build process, with one change: you must pass the \mono{--enable-mpi} flag to our configure script to cause HMMER to be built with MPI support:

\vspace{1ex}
\user{\% ./configure {-}{-}prefix=/your/install/path {-}{-}enable-mpi}\\
\user{\% make}
\vspace{1ex}

\Hmmclient does not require MPI support, as it is a single-threaded program that communicates with a server via sockets.  If you only want to use \hmmclient to send searches to an existing server, any installation of HMMER, including ones from package managers or Linux distributions, should provide a working version of \hmmclient.

\chapter{Usage}

\section{Hmmserver}

\section{Hmmclient}

\chapter{For those converting from an hmmpgmd installation}

\chapter{Desgn of hmmserver}
\section{Parallelization and Performance}
\section{Sharding}
\section{Client-Server Interface}

\begin{adjustwidth}{}{-1in}   
\chapter{Manual Pages Related to the Server}

\end{adjustwidth}

\chapter{Acknowledgments}
Simon Potter of the European Bioinformatics Institute was of great help in understanding the daemon's interactions with the EBI's web servers.  We would also like to thank all of the organizations that have supported the development of HMMER, as well as all of the individuals who have contributed to it. In particular, Washington University, the National Institutes of Health, Monsanto, the Howard Hughes Medical Institute, and Harvard University have been major supporters of this work.  For a more thorough set of acknowledgments that includes a discussion of HMMER's history, please see the \underline{HMMER User's Guide}.

\label{manualend}

% To create distributable/gitted 'distilled.bib' from lab's bibtex dbs:
%   # uncomment the {master,lab,books}
%   pdflatex main
%   bibdistill main.aux > distilled.bib
%   # restore the {distilled} 
% 
\nobibliography{distilled}
%\nobibliography{master,lab,books}

\end{document}



